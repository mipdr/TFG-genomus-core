\thispagestyle{empty}

\begin{center}
{\large\bfseries \titulo \\ \subtitulo }\\
\end{center}
\begin{center}
\autor\\
\end{center}

%\vspace{0.7cm}

\vspace{0.5cm}
\noindent\textbf{Palabras clave}: \textit{software libre, gestión de proyecto, SCRUM, metodologías ágiles, programación funcional, lenguajes de programación, parser, intérprete, programación musical, C++, javascript, NodeJS, NodeJS add-ons, DevOps}
\vspace{0.7cm}

\noindent\textbf{Resumen}\\
Este documento refleja el trabajo realizado por el autor sobre el proyecto GenoMus, consistente en un rediseño e implementación de su motor de cómputo funcional en forma de biblioteca de C++ y de la utilización de esta en un módulo nativo de NodeJS para exponer las funcionalidades al entorno de ejecución de Javascript para su futura integración con el software GenoMus. El software implementado incluye también una interfaz de línea de comandos capaz de interpretar y evaluar expresiones de GenoMus.

\cleardoublepage

\begin{center}
	{\large\bfseries Contributions to GenoMus\\ Redesign of a funcional computing engine for musical structures}\\
\end{center}
\begin{center}
	\autor\\
\end{center}
\vspace{0.5cm}
\noindent\textbf{Keywords}: \textit{open source, project management, SCRUM, agile, functional programming, programming languages, parser, interpreter, musical programming, NodeJS, NodeJS add-ons, DevOps }
\vspace{0.7cm}

\noindent\textbf{Abstract}\\
This document reflects on the work done by the author for the GenoMus project, consisting on a redesign and implemnetation of its functional computational model as a C++ library and the creation of a NodeJS native module that exposes the functionalities of this library to the Javascript runtime. The software also includes a CLI that can interpret and evaluate GenoMus expressions.


\cleardoublepage

\thispagestyle{empty}

\noindent\rule[-1ex]{\textwidth}{2pt}\\[4.5ex]

D. \textbf{Miguel Molina Solana}, Profesor del Departamento de Ciencias de la Computación e Inteligencia Artificial de la Universidad de Granada.

D. \textbf{José López Montes}, Catedrático de Tecnología Musical del Real Conservatorio Superior de Música Victoria Eugenia de Granada.

\vspace{0.5cm}

\textbf{Informo:}

\vspace{0.5cm}

Que el presente trabajo, titulado \textit{\textbf{\titulo: \subtitulo}},
ha sido realizado bajo mi supervisión por \textbf{\autor}, y autorizo la defensa de dicho trabajo ante el tribunal que corresponda.

\vspace{0.5cm}

Y para que conste, expiden y firman el presente informe en Granada a Julio de 2022.

\vspace{1cm}

\textbf{El/la director(a)/es: }

\vspace{5cm}

\noindent \textbf{(Miguel Molina Solana)} \hspace{5cm} \textbf{(José López Montes)}

\chapter*{Agradecimientos}

\begin{flushright}
    A \textbf{Miguel e Irene}, mis padres, porque este trabajo no habría sido posible sin su trabajo y apoyo a lo largo de los años.
\end{flushright}

\begin{flushright}
    A \textbf{José López Montes}, por su gran labor como profesor y por introducirme al proyecto GenoMus.
\end{flushright}

\begin{flushright}
    A \textbf{Miguel Molina Solana}, por su rol como mentor en este proyecto y por los conocimientos compartidos a lo largo de este.
\end{flushright}

