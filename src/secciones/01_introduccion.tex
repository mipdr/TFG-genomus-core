\chapter{Introducción}

Este proyecto consiste en una propuesta de implementación del marco teórico del proyecto de software libre GenoMus\cite{GenoMus}, que tiene el objetivo de contribuir a la calidad y proyección del software del proyecto. La totalidad del código que conforma la implementación presentada en esta memoria está liberada bajo la licencia MIT\cite{mit} y se encuentra disponible en los repositorios públicos citados en la bibliografía.

\section{GenoMus}

El proyecto GenoMus (\textsf{/xe'nomus/} o \textsf{/xeno'mus/}) iniciado por José López Montes en 2013 expone un marco teórico-práctico en el que las estructuras musicales que un oyente percibe a nivel cognitivo se rigen por estructuras anidadas de datos musicales representables mediante gramáticas no deterministas. Propone un modelo de datos y un modelo de cómputo a través de los cuales se pueden crear instancias de datos musicales con sentido para el oyente, además de analizar la estructura de instancias musicales existentes. Cuenta con un modelo de datos bio-inspirado basado en las diferentes transformaciones que sufre el material genético en las células y un modelo de datos cómputo inspirado por el paradigma de programación funcional. Previo al comienzo del proyecto que describe esta memoria, GenoMus ha contado y cuenta con una propuesta de implementación en forma de prototipo\cite{GenoMus} desarrollada por José López Montes. Este prototipo ha llevado el marco teórico a la realidad y ha posibilitado el uso del modelo para su principal propósito: la Composición Asistida por Computadora (CAC).

\section{Motivación}\label{motivacion}

Tras conocer el proyecto GenoMus a finales de 2021 e interesado por la teoría y la práctica de la Composición Asistida por Computadora o CAC, el autor propone realizar una serie de contribuciones técnicas al proyecto con el objetivo de aumentar la calidad de GenoMus como proyecto de software libre, la primera de las cuales queda enmarcada en este proyecto de fin de grado.