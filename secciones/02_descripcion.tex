\chapter{Descripción del problema}

El diseño de la implementación realizada busca dar la talla en los dos siguientes rasgos principales:

 - El código debe poder ser entendido y mantenido por un programador amateur. Al pertenecer al campo de las humanidades digitales tenemos que tener en cuenta a los usuarios con perfiles no técnicos. Por esto la legibilidad, simplicidad y declaratividad del código son prioritarias.
 - El programa debe ser lo más eficiente posible en regiones de cómputo intensivo. Aspiramos a que el límite en el volumen o complejidad de datos que el programa genera no repercuta sobre la capacidad de creación del usuario ni sobre la usabilidad del software.

