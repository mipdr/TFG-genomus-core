\chapter{Análisis del problema}

\section{Análisis del prototipo}
 
Como se ha mencionado previamente en el apartado de descripción del problema, el objetivo principal del proyecto es dotar a GenoMus de una implementación de carácter profesional a prueba de futuro que sea mantenible por perfiles no técnicos.

El prototipo de GenoMus está implementado en javascript puro para ejecución sobre NodeJS. La sintaxis utilizada busca la utilización del "dialecto" funcional de javascript. Tanto el modelo de representación de datos como el modelo de cómputo siguen generalmente patrones de diseño de programación funcional. 

\subsection{Ventajas del planteamiento funcional del código en JS.}
Es bonito

\subsection{Placebos del JS funcional.} \ \
Javascript nos proporciona azúcar sintáctico sobre la creación de objetos dinámicos que nos da la posiblidad de implementar entidades como clausuras o funciones de orden superior. Sin embargo estas construcciones que tan bien nos entran por los ojos no están acompañadas de un respaldo acorde en el entorno de ejecución de js. Ninguno de los principales motores de js incluye algún tipo de análisis de los objetos funcionales para su optimización en grupo, como procedimientos de composición de funciones o un motor de reducción de grafos de dependencias computacionales.

La composición de funciones es la característica cuya ausencia puede haber condicionado el desarrollo del prototipo. El prototipo implementa un procedimiento de almacenamiento de árboles funcionales a través de cadenas de texto. Las diferentes construcciones funcionales se almacenan como código javascript, de cuya ejecución se obtiene la evaluación del árbol. Este tipo de metaprogramación aumenta el cómputo necesario para evaluar árboles de funciones, ya que se introduce una etapa de construcción de código y una etapa de parseo de este, validación y posterior ejecución.
